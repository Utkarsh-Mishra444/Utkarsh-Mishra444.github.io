\documentclass[a4paper,10pt]{report}
\usepackage[margin=0in ]{geometry}
\usepackage{latexsym}
\usepackage{float}
\usepackage[empty]{fullpage}
\usepackage{wrapfig}
\usepackage{lipsum}
\usepackage{fontawesome5}
\usepackage{enumitem}
\usepackage{tabu}
\usepackage{textcomp}
\usepackage{titlesec}
\usepackage{marvosym}
\usepackage[usenames,dvipsnames]{color}
\usepackage{verbatim}
\usepackage{enumitem}
\usepackage[hidelinks]{hyperref}
\usepackage{fancyhdr}
\usepackage{multicol}
\usepackage{amsmath}
\usepackage{graphicx}
\setlength{\multicolsep}{0pt} 
\pagestyle{fancy}
\fancyhf{} % clear all header and footer fields
\fancyfoot{}
\renewcommand{\headrulewidth}{0pt}
\renewcommand{\footrulewidth}{0pt}
\definecolor{lightblue}{RGB}{50,72,240} 
% Adjust margins
\addtolength{\oddsidemargin}{-0.5in}
\addtolength{\evensidemargin}{-0.5in}
\addtolength{\textwidth}{1.0in}
\addtolength{\topmargin}{-0.75in}
\addtolength{\textheight}{1.3in}
% \addtolength{\bottommargin}{-0.3in}

\urlstyle{same}

%\raggedbottom
% \raggedright
\setlength{\tabcolsep}{0in}

% Sections formatting
\titleformat{\section}{
  \vspace{-9pt}\scshape\raggedright\large\textbf
}{}{0em}{}[\color{black}\titlerule \vspace{-4pt}]

%-------------------------
% Custom commands
\newcommand{\resumeItem}[2]{
  \item\small{
    \textbf{#1}{: #2 \vspace{-2pt}}
  }
}
\newcommand{\resumeSubheading}[4]{
  \vspace{-1pt}\item
    \begin{tabular*}{0.97\textwidth}[t]{l@{\extracolsep{\fill}}r}
      \textbf{#1} & #2 \\
      \textit{\small#3} & \textit{\small #4} \\
    \end{tabular*}\vspace{-6pt}
}

\newcommand{\resumeSubItem}[2]{\resumeItem{#1}{#2}\vspace{-4pt}}

\renewcommand{\labelitemii}{$\circ$}

\newcommand{\resumeSubHeadingListStart}{\begin{itemize}[leftmargin=*]}
\newcommand{\resumeSubHeadingListEnd}{\end{itemize}}
\newcommand{\resumeItemListStart}{\begin{itemize}}
\newcommand{\resumeItemListEnd}{\end{itemize}\vspace{-6pt}}

%-------------------------------------------
%%%%%%  CV STARTS HERE  %%%%%%%%%%%

\begin{document}

\centering

\vspace{1pt}

\begin{tabular}[h!]{c@{\hskip 1.2cm}r}

\textbf{\LARGE \href{https://utkarsh-mishra444.github.io/}{Utkarsh Mishra}} & 
\\

\href{mailto:utkarshmishra@tamu.edu}{M.S. CE @ Texas A\&M, utkarshmishra@tamu.edu} & 
\\

\href{https://github.com/Utkarsh-Mishra444}{\faGithub Github}  \href{https://www.linkedin.com/in/utkarsh-mishra-663b5a213/}{\faLinkedin LinkedIn} \href{https://utkarsh-mishra444.github.io/}{ \faUser Website} & 
\end{tabular}

\vspace{-1pt}
%-----------EDUCATION-----------------
\vspace{-5pt}
\section{Education}

\noindent
\begin{itemize}[leftmargin=0.5in, rightmargin=0.5in]

\vspace{-6pt}

\item \textbf{Texas A\&M University (TAMU)} \hfill \textbf{Starting Fall 2025} \\[6pt]
\vspace{-5pt}
    M.S. Computer Engineering \\
    Recipient of the ECEN Merit Scholarship
\vspace{-3pt}

\item \textbf{Indian Institute of Technology Gandhinagar (IIT-GN)} \hfill \textbf{2020 – 2024} \\[6pt]
\vspace{-4pt}
    B.Tech in Electrical Engineering with Minor in Computer Science \\CGPA: 8.79/10
\end{itemize}
\vspace{-12pt}
\section{Interest Areas}
\begin{flushleft}
Vision Language Models, Agentic Systems,     Graph Neural Networks, Transformers, Diffusion Models, Gaussian Splatting , Neural Radiance Fields, Synthetic Data Generation  
\end{flushleft}
\vspace{-12pt}
%-----------EXPERIENCE-----------------
\section{Industrial Work Experience}
  \resumeSubHeadingListStart
    \resumeSubheading
      {System Analyst @ HDFC Bank Ltd}{(Jul 2024 -- Present)}{}{}
       \vspace{-12pt}
       \begin{itemize}
        \item Member of the CMDB (Configuration Management Database) team, responsible for onboarding and Discovery of organizational IT infrastructure to enable automatic population of configuration data in the ServiceNow CMDB. \\
        \vspace{-2pt}
        \item Coordinated Discovery configuration implementations with application teams across the organization, identifying and addressing Discovery errors to ensure CMDB reliability, integrity, and compliance with regulatory standards.
        \vspace{-15pt}
        \item Developed automation solutions: TypeScript-based Office Scripts for data analysis and Bash/PowerShell scripts that automate testing and troubleshooting of Discovery errors.\\ 
     \end{itemize}
     \vspace{-9pt}

  \resumeSubheading
      {System Analyst Intern @ HDFC Bank Ltd}{(May 2023 -- Jul 2023)}{}{}
       \vspace{-12pt}
       \begin{itemize}
        \item Member of the CMDB team that facilitated the transition from CA ServiceDesk R17 to ServiceNow; set up Discovery, troubleshot device and server errors, and developed automation tools to extend Discovery capabilities.
        \vspace{-15pt}
        \item Identified and resolved connection, authentication, and other issues across 1,000+ devices; analyzed large datasets to detect and report CMDB data gaps, ensuring audit compliance.   \\
        \vspace{-2pt}
        \item Received pre-placement offer from the company based on excellent performance. 
     \end{itemize}
     \vspace{-9pt}
     
\resumeItemListEnd
\section{Research Experience}
  \resumeSubHeadingListStart
       \resumeSubheading
      {Research Internship @ University of Illinois, Urbana-Champaign}{(Jan 2025 -- Present)}{Mentor: Prof. Svetlana Lazebnik, Prof. Unnat Jain,  Dwip Dalal (PhD Student)}{}
       \begin{itemize}
        \item \textbf{Constructive Distortion: Multimodal LLMs with Query‑Aware Image Warping.} \hspace{25pt} \textcolor{lightblue}{\href{https://dwipddalal.github.io/Attwarp/}{(Project Page)}}
        \\ Designed an image warping algorithm “Att-warp” that increases semantically relevant content of an image using an MLLM’s own query conditioned attention map over the image. Achieved \textbf{SOTA} performance on LLaVA and Qwen-VL on TextVQA, GQA, MMMU, POPE, and DocVQA datasets. \hyperref[sec: Att-Warp]{\textbf{\textcolor{lightblue}{[2]}}}   

        \item \textbf{Agentic Self-Improving Pipeline for Street View Navigation} \\ Architected an autonomous navigation system that visually analyzes Street View imagery to determine navigation decisions using purely visual information, incorporating memory-based reasoning, robust error handling, and iterative self-improvement through analysis of past navigation errors.

 \end{itemize}
 \vspace{-8}
       \resumeSubheading
      {Research Assitant @ IIT Gandhinagar}{(August 2023 -- Feb 2025)}{Mentor: Prof. Shanmuganathan Raman, Prof. Ravi Hegde}{}
      %\vspace{-12pt}
       \begin{itemize}
        \item \textbf{Diffusion Transformer for 3D Point Cloud Denoising}  \\ Developed a transformer-based model for point cloud denoising building upon P2P-Bridge method by using multi-
resolution hash encoding, providing an alternate to PointNet based networks. Achieved SOTA performance on PCNet and PUNet datasets at 1\% noise levels. \hyperref[sec: diffusion-transformer]{\textbf{\textcolor{lightblue}{[1]}}}   
        \item \textbf{Denoising Gaussian Splatting For 3D Scene Reconstruction} \hspace{136pt} \textcolor{lightblue}{\href{https://github.com/Utkarsh-Mishra444/Denoising-Gaussian-Splatting}{(Repository)}}
        \\ Extended 3D Gaussian Splatting by applying denoising techniques (DBSCAN, point-wise distance pairing) to the input COLMAP sparse point cloud to reduce visual artifacts. Developed a novel regularization methods to mitigate overfitting. 
        \item \textbf{Human Pose Classification using Spatial-Temporal Graph Neural Network} \hspace{25pt} \textcolor{lightblue}{\href{https://drive.google.com/file/d/1eB51xk6ZKn7HR4SFfv9kwp1dmNQnQqGG/view?usp=sharing}{(Poster)}}\textcolor{lightblue}{\href{https://github.com/Utkarsh-Mishra444/Human-Pose-GNN}{(Repository)}}
        \\ Developed a graph-neural-network–based human-pose classifier. Utilized OpenPose, AlphaPose, and YOLOv7 for keypoint detection and pose-graph construction. Implemented a hybrid Graph Convolutional Network (GCN) – LSTM architecture. Trained and evaluated the model on the MPOSE-2021 dataset.
     \end{itemize}
     
\resumeItemListEnd
\vspace{-10pt}
\section{Publications}
\hspace{-20pt}\textbf{[1] Transformer Augmented Multi‐Resolution Hash Encoding in Diffusion Model for 3D Point Cloud} \label{sec: diffusion-transformer} \\

\hspace{-444pt}\textbf{Denoising}


\hspace{-144pt} \textit{Seema Kumari, \textbf{Utkarsh Mishra}, Srimanta Mandal, Shanmuganathan Raman} \label{sec: Diff_ldr_hdr}
 
\hspace{-128pt} \texttt{Accepted at the International Conference on Image Processing (ICIP'25)}

\hspace{-113pt}\textbf{[2] Constructive Distortion: Multimodal LLMs with Query‑Aware Image Warping} \\

\hspace{-74pt} \textit{Dwip Dalal, Gautam Vashishtha, \textbf{Utkarsh Mishra}, ... Svetlana Lazebnik, Heng Ji, Unnat Jain} \label{sec: Att-Warp}
 
\hspace{-360pt} \texttt{Under review at NeurIPS'25}

\vspace{-4pt}
%------------- Projects -------
\section{Projects}
  \resumeSubHeadingListStart

    \resumeSubheading
      {Synthetic Data Generation for Machine Learning}{(Aug 2023 - May 2024)}{Mentor: Prof. Shanmuganathan Raman, IIT Gandhinagar}
      {\textcolor{lightblue}{\href{https://drive.google.com/file/d/1EdvRO9tcQluWlHwwh-OfD8RuvAoGLDWt/view?usp=sharing}{(Poster)}}\textcolor{lightblue}{\href{https://github.com/Utkarsh-Mishra444/Syn-Gen}{(Repository)}}}{}
      %\vspace{-12pt}
       \begin{itemize}
        \item Generated high-quality synthetic images using StyleGAN-XL and Stable Diffusion tailored to the CIFAR-10 dataset, producing over 90,000 images to augment training data.
        \vspace{-2pt}
        \item Engineered advanced prompts by extracting visual attributes from the Visual Genome dataset, aligning class labels with WordNet synsets, clustering visual attributes with GloVe embeddings, and sampling from clusters to increase diversity of attributes. Further prompt enhancements using Gemma 2B-it.
        \vspace{-2pt}
        \item Conducted comparative analysis of classifiers (e.g. ResNet-34) trained on varying ratios of synthetic to real data through extensive experimentation on different types of synthetic data. Analyzed effects on classifier accuracy and confusion matrix. 

     \end{itemize}


\vspace{-6pt}

    \resumeSubheading
      {Japanese Language Learning App}{(May 2024 - Jun 2024)}{Personal Project}
      {\textcolor{lightblue}{\href{https://github.com/Utkarsh-Mishra444/Immerse}{(Repository)}}}{}
      %\vspace{-12pt}
       \begin{itemize}
        \item Developed a native Android application in Kotlin using Jetpack Compose with an integrated open-source JMDICT dictionary to enhance Japanese language learning by providing contextual word usage from YouTube videos. \\
        \vspace{-2pt}
        \item Implemented natural language processing using Kuromoji to handle Japanese word conjugations by converting words into base forms, improving search accuracy and usability. Utilized the YouTube Transcript API with Python integration via Chaquopy for transcript processing.
     \end{itemize}


\vspace{-6pt}

    \resumeSubheading
      {GAN Inversion for Latent Space Analysis}{(Jan 2024 - May 2024)}{ES 413 Deep Learning, IIT Gandhinagar}
      {\textcolor{lightblue}{\href{https://github.com/Utkarsh-Mishra444/Gan-Inversion}{(Repository)}}}{}
      %\vspace{-12pt}
       \begin{itemize}
        \item Used GAN inversion on StyleGAN to invert images of cars into their latent representation and analyzed effect of object rotation on latent representation and generated novel views. 
        \vspace{-2pt}
        \item Identified and analyzed issues such as imperfect reconstruction, editability vs reconstruction error trade-off, the influence of the GAN's training data on outputs, background effects, and visual artifacts from latent manipulation.
     \end{itemize}


\vspace{-6pt}

    \resumeSubheading
      {Assorted PID Controller-Based Mechanisms}{(Jan 2024 - Mar 2024)}{Personal Project}
      {}{}
      %\vspace{-12pt}
       \begin{itemize}
        \item Utilized an Arduino to develop three systems: a ball balancing on a beam, an inverted pendulum, and a reaction wheel, all based on PID control.
        \vspace{-2pt}

     \end{itemize}


\vspace{-6pt}

    \resumeSubheading
      {Wearable Device for Real Time Sign Language Recognition}{(Apr 2023)}{ES 333 Microprocessors and Embedded Systems, IIT Gandhinagar}
      {\textcolor{lightblue}{\href{https://github.com/Utkarsh-Mishra444/Sign-Recog-Glove}{(Repository)}}}{}
      %\vspace{-12pt}
       \begin{itemize}
        \item Worked in a team of 4 to develop a wearable device for sign language recognition using a STM32 Nucleo microcontroller and flex sensors for data acquisition. 
        \vspace{-2pt}
        \item Implemented USB CDC protocol to communicate sensor data to host device running Scikit-learn based MLP classifier used to detect gestures.
        \vspace{-2pt}

     \end{itemize}


\vspace{-6pt}

    \resumeSubheading
      {Experimental Analysis of Stokes-Einstein \& Stokes-Einstein-Debye Equations}{(Aug 2022 - Nov 2022)}{BS 191 Matter and Energy Laboratory, IIT Gandhinagar}
      {\textcolor{lightblue}{\href{https://drive.google.com/file/d/1bTYEfIJr_yOuy1eHvitEQpjc5D6bzoEW/view?usp=sharing}{(Report)}}}{}
      %\vspace{-12pt}
       \begin{itemize}
        \item Led a team of five members to undertake the verification of the Stokes-Einstein and Stokes-Einstein-Debye Equations. Performed project planning, apparatus procurement, and project execution.
        \vspace{-2pt}
        \item Managed team effort , including a critical 48-hour reaction phase, organizing team members' schedules for continuous supervision, leading to successful synthesis and analysis of polystyrene colloidal particles.
        \vspace{-2pt}
        \item Designed a custom stretching device using Autodesk Inventor critical for processing of particles. Developed code for motion tracking of particle images obtained through microscopy.
     \end{itemize}


\resumeItemListEnd


\vspace{1pt}
% \section{Teaching Experience \hfill \small(Certificate)}
\section{Teaching Experience}

\resumeSubHeadingListStart
\resumeSubheading{Academic Discussion Hour Mentor, Engineering Graphic} {(Feb 2022 - Apr 2022)}
{}{}
\vspace{-12pt}
\begin{itemize}
\item Selected as one of the four student mentors to help teach the course ES101 Engineering Graphics in tandem with regular classes. Selected based on past academic excellence in the course.

\vspace{-3pt}
\item Held weekly sessions to clear student queries, deliver personalized mentoring to improve learning outcomes.
\end{itemize}


\resumeSubHeadingListEnd



\section{Relevant Courses}
\vspace{-2pt}
 % \begin{multicols}{2}
 \begin{itemize}[leftmargin = *,itemsep=-3pt]

    Deep Learning, Computation and Cognition, Probability and Random Processes, Signals Systems and Networks, Data Structures and Algorithms, Linear Algebra and Single Variable Calculus, Multivariable Calculus and Complex Analysis, Ordinary Differential Equations, Probability Statistics and Numerical Methods, Digital Signal Processing, Control Theory, Digital Systems, Microprocessors and Embedded Systems, Computer Organization and Architecture, Operating Systems, Databases
    

\end{itemize}

%\vspace{-15pt}

\section{Technical Skills Summary}
 \resumeSubHeadingListStart
 \resumeSubItem{Programming Languages}
 {Java , Python, C, C++, Kotlin, Verilog, Matlab.}
 \resumeSubItem{Platforms \& Frameworks}{PyTorch, Scikit-learn, Google Cloud, GitHub, Linux}
  \resumeSubItem{Architectures \& Technologies}{Vision Language Models, Diffusion Models, VAEs, GANs , LSTM, Transformers, Neural Radiance Fields (NeRFs) , 3D Gaussian Splatting, Graph Neural Networks, Microcontrollers (Arduino, STM32 Nucleo, NodeMCU) }
 
 \resumeSubHeadingListEnd


\end{document}
